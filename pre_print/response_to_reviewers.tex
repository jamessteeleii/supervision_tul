% Options for packages loaded elsewhere
\PassOptionsToPackage{unicode}{hyperref}
\PassOptionsToPackage{hyphens}{url}
\PassOptionsToPackage{dvipsnames,svgnames,x11names}{xcolor}
%
\documentclass[
  letterpaper,
  DIV=11,
  numbers=noendperiod]{scrartcl}

\usepackage{amsmath,amssymb}
\usepackage{iftex}
\ifPDFTeX
  \usepackage[T1]{fontenc}
  \usepackage[utf8]{inputenc}
  \usepackage{textcomp} % provide euro and other symbols
\else % if luatex or xetex
  \usepackage{unicode-math}
  \defaultfontfeatures{Scale=MatchLowercase}
  \defaultfontfeatures[\rmfamily]{Ligatures=TeX,Scale=1}
\fi
\usepackage{lmodern}
\ifPDFTeX\else  
    % xetex/luatex font selection
\fi
% Use upquote if available, for straight quotes in verbatim environments
\IfFileExists{upquote.sty}{\usepackage{upquote}}{}
\IfFileExists{microtype.sty}{% use microtype if available
  \usepackage[]{microtype}
  \UseMicrotypeSet[protrusion]{basicmath} % disable protrusion for tt fonts
}{}
\makeatletter
\@ifundefined{KOMAClassName}{% if non-KOMA class
  \IfFileExists{parskip.sty}{%
    \usepackage{parskip}
  }{% else
    \setlength{\parindent}{0pt}
    \setlength{\parskip}{6pt plus 2pt minus 1pt}}
}{% if KOMA class
  \KOMAoptions{parskip=half}}
\makeatother
\usepackage{xcolor}
\setlength{\emergencystretch}{3em} % prevent overfull lines
\setcounter{secnumdepth}{-\maxdimen} % remove section numbering
% Make \paragraph and \subparagraph free-standing
\ifx\paragraph\undefined\else
  \let\oldparagraph\paragraph
  \renewcommand{\paragraph}[1]{\oldparagraph{#1}\mbox{}}
\fi
\ifx\subparagraph\undefined\else
  \let\oldsubparagraph\subparagraph
  \renewcommand{\subparagraph}[1]{\oldsubparagraph{#1}\mbox{}}
\fi


\providecommand{\tightlist}{%
  \setlength{\itemsep}{0pt}\setlength{\parskip}{0pt}}\usepackage{longtable,booktabs,array}
\usepackage{calc} % for calculating minipage widths
% Correct order of tables after \paragraph or \subparagraph
\usepackage{etoolbox}
\makeatletter
\patchcmd\longtable{\par}{\if@noskipsec\mbox{}\fi\par}{}{}
\makeatother
% Allow footnotes in longtable head/foot
\IfFileExists{footnotehyper.sty}{\usepackage{footnotehyper}}{\usepackage{footnote}}
\makesavenoteenv{longtable}
\usepackage{graphicx}
\makeatletter
\def\maxwidth{\ifdim\Gin@nat@width>\linewidth\linewidth\else\Gin@nat@width\fi}
\def\maxheight{\ifdim\Gin@nat@height>\textheight\textheight\else\Gin@nat@height\fi}
\makeatother
% Scale images if necessary, so that they will not overflow the page
% margins by default, and it is still possible to overwrite the defaults
% using explicit options in \includegraphics[width, height, ...]{}
\setkeys{Gin}{width=\maxwidth,height=\maxheight,keepaspectratio}
% Set default figure placement to htbp
\makeatletter
\def\fps@figure{htbp}
\makeatother
\newlength{\cslhangindent}
\setlength{\cslhangindent}{1.5em}
\newlength{\csllabelwidth}
\setlength{\csllabelwidth}{3em}
\newlength{\cslentryspacingunit} % times entry-spacing
\setlength{\cslentryspacingunit}{\parskip}
\newenvironment{CSLReferences}[2] % #1 hanging-ident, #2 entry spacing
 {% don't indent paragraphs
  \setlength{\parindent}{0pt}
  % turn on hanging indent if param 1 is 1
  \ifodd #1
  \let\oldpar\par
  \def\par{\hangindent=\cslhangindent\oldpar}
  \fi
  % set entry spacing
  \setlength{\parskip}{#2\cslentryspacingunit}
 }%
 {}
\usepackage{calc}
\newcommand{\CSLBlock}[1]{#1\hfill\break}
\newcommand{\CSLLeftMargin}[1]{\parbox[t]{\csllabelwidth}{#1}}
\newcommand{\CSLRightInline}[1]{\parbox[t]{\linewidth - \csllabelwidth}{#1}\break}
\newcommand{\CSLIndent}[1]{\hspace{\cslhangindent}#1}

\KOMAoption{captions}{tableheading}
\makeatletter
\makeatother
\makeatletter
\makeatother
\makeatletter
\@ifpackageloaded{caption}{}{\usepackage{caption}}
\AtBeginDocument{%
\ifdefined\contentsname
  \renewcommand*\contentsname{Table of contents}
\else
  \newcommand\contentsname{Table of contents}
\fi
\ifdefined\listfigurename
  \renewcommand*\listfigurename{List of Figures}
\else
  \newcommand\listfigurename{List of Figures}
\fi
\ifdefined\listtablename
  \renewcommand*\listtablename{List of Tables}
\else
  \newcommand\listtablename{List of Tables}
\fi
\ifdefined\figurename
  \renewcommand*\figurename{Figure}
\else
  \newcommand\figurename{Figure}
\fi
\ifdefined\tablename
  \renewcommand*\tablename{Table}
\else
  \newcommand\tablename{Table}
\fi
}
\@ifpackageloaded{float}{}{\usepackage{float}}
\floatstyle{ruled}
\@ifundefined{c@chapter}{\newfloat{codelisting}{h}{lop}}{\newfloat{codelisting}{h}{lop}[chapter]}
\floatname{codelisting}{Listing}
\newcommand*\listoflistings{\listof{codelisting}{List of Listings}}
\makeatother
\makeatletter
\@ifpackageloaded{caption}{}{\usepackage{caption}}
\@ifpackageloaded{subcaption}{}{\usepackage{subcaption}}
\makeatother
\makeatletter
\@ifpackageloaded{tcolorbox}{}{\usepackage[skins,breakable]{tcolorbox}}
\makeatother
\makeatletter
\@ifundefined{shadecolor}{\definecolor{shadecolor}{rgb}{.97, .97, .97}}
\makeatother
\makeatletter
\makeatother
\makeatletter
\makeatother
\ifLuaTeX
  \usepackage{selnolig}  % disable illegal ligatures
\fi
\IfFileExists{bookmark.sty}{\usepackage{bookmark}}{\usepackage{hyperref}}
\IfFileExists{xurl.sty}{\usepackage{xurl}}{} % add URL line breaks if available
\urlstyle{same} % disable monospaced font for URLs
\hypersetup{
  pdftitle={Response to Reviewers},
  colorlinks=true,
  linkcolor={blue},
  filecolor={Maroon},
  citecolor={Blue},
  urlcolor={Blue},
  pdfcreator={LaTeX via pandoc}}

\title{Response to Reviewers}
\usepackage{etoolbox}
\makeatletter
\providecommand{\subtitle}[1]{% add subtitle to \maketitle
  \apptocmd{\@title}{\par {\large #1 \par}}{}{}
}
\makeatother
\subtitle{The effects of supervision on effort during resistance
training: A Bayesian analysis of prior data and an experimental study of
private strength clinic members}
\author{}
\date{}

\begin{document}
\maketitle
\ifdefined\Shaded\renewenvironment{Shaded}{\begin{tcolorbox}[boxrule=0pt, breakable, interior hidden, sharp corners, enhanced, frame hidden, borderline west={3pt}{0pt}{shadecolor}]}{\end{tcolorbox}}\fi

Dear Professor Gruet,

Many thanks for considering our manuscript and placing it under further
review with \emph{Peer Community in Health and Movement Sciences}.

We apologise sincerely for our delay in addressing these and
resubmitting. Unfortunately other commitments and workloads had
prevented us from returning to this. We have attempted to address or
rebut the additional comments provided by the reviewers.

Again, we would like to thank all involved in the process of reviewing
this manuscript and look forward to the next round of reviews in
considering our revisions.

Many thanks

The authors

\hypertarget{reviewer-comments-and-author-responses}{%
\section{Reviewer Comments and Author
Responses}\label{reviewer-comments-and-author-responses}}

\hypertarget{reviewer-1---david-clarke}{%
\subsection{Reviewer 1 - David Clarke}\label{reviewer-1---david-clarke}}

\hypertarget{reviewer-comment}{%
\subsubsection{Reviewer Comment}\label{reviewer-comment}}

Title - Suggest changing ``upon'' to ``on.''

\hypertarget{author-response}{%
\subsubsection{Author Response}\label{author-response}}

We have made this change.

\hypertarget{reviewer-comment-1}{%
\subsubsection{Reviewer Comment}\label{reviewer-comment-1}}

Abstract - TUL, RPE, and RPD are measures of effort, not exercise
performance. There are several instances in which ``exercise
performance'' is used in place of ``effort,'' wherein only the latter
should be used. Be consistent with the constructs and operational
definitions.

\hypertarget{author-response-1}{%
\subsubsection{Author Response}\label{author-response-1}}

In the present study exercise performance as TUL is used as an
operationalisation to draw inferences about effort. As such, in places
where we are drawing broader inferences about the concept of effort as
informed by the exercise performance we use ``effort'', but in others we
use more accurately to the operationalisation ``exercise performance''
or ``TUL''. We have added the following the abstract to clarify this:

\emph{``\ldots exercise performance, measured as time under load (TUL)
where a longer TUL at a given load is indicative of greater
effort\ldots{}''}

We have also added to the introduction the text in parentheses at the
end of this sentence to clarify this conceptually:

\emph{``Effort is conceptualised at the relation of task demands to the
capacity to meet those demands (Steele 2020) and so in RT is determined
by both the load utilised and also the proximity to momentary failure
due to the fatigue (i.e., reduction in capacity) that occurs with
continued performance at a given load (i.e., with more repetitions or a
longer time under load).''}

And clarified in the methods:

\emph{``The TUL performed (where a longer TUL was interpreted as
indicating greater effort), and ratings of perceptions of effort (RPE)
and discomfort (RPD) were recorded and compared between both
conditions.''}

\hypertarget{reviewer-comment-2}{%
\subsubsection{Reviewer Comment}\label{reviewer-comment-2}}

Abstract - Second sentence: ``We investigated supervision's role in
effort\ldots{}'' -- ``in'' should be ``on''

\hypertarget{author-response-2}{%
\subsubsection{Author Response}\label{author-response-2}}

We have made this change.

\hypertarget{reviewer-comment-3}{%
\subsubsection{Reviewer Comment}\label{reviewer-comment-3}}

The Introduction describes the existing knowledge of the effect of
supervision on resistance training outcomes, specifically strength, and
summarizes the study's design and methods. While the essential content
is present, it could be more clearly discussed. I suggest reorganizing
the introduction as follows: • The first paragraph is good as is. • I
would present the hypothesized mechanisms together up front, i.e.,
trainers enforce load progression and help motivate the exercisers to
apply the proper amount of effort (i.e., closer proximity to momentary
muscular failure). • I would then discuss the evidence for each
hypothesized mechanism, which is mostly indirect. • The second-to-last
paragraph should summarize the existing gaps in knowledge, i.e., the
data concerning supervision effects are still limited, especially in
ecological settings, and the potential mechanisms by which supervision
might enhance strength are likewise unclear.\\
• The final paragraph should more explicitly state the study purpose and
along with a hypothesis. For example, ``The purpose of this study was to
examine the effects of supervision on client effort in resistance
training in a more ecological setting. We hypothesized that \ldots{} I
would then introduce the methods as the authors have done by stating
that ``We tested the hypothesis by \ldots{}''

\hypertarget{author-response-3}{%
\subsubsection{Author Response}\label{author-response-3}}

We thank the reviewer for their suggested revision to the introduction.
We have however opted not to make any major rewrites to it as we do feel
the major rationale is presented clearly already, and we have not added
any hypothesis as our analyses were not aimed at any kind of statistical
hypothesis testing but instead exploratory and descriptive parameter
estimation. Lastly, given the suggested restructure is largely one of
preference, this is somewhat of a pragmatic choice on our part to not
make changes because of available time for making revisions given other
workload demands currently (and that we're already pretty late with
making revisions in response to this second round of review - our
apologies!).

\hypertarget{reviewer-comment-4}{%
\subsubsection{Reviewer Comment}\label{reviewer-comment-4}}

Minor comments on the writing: There were several errors and suboptimal
wording choices present. • The word ``recent'' (typically followed by
``meta-analysis'') is repetitively used throughout. Just state the
result and cite the paper. For example, the second sentence, ``In a
recent systematic review and meta-analysis from Fisher et al.~(2022)
there was a moderate standardised mean effect (0.40 {[}95\%CI: 0.06,
0.74{]}) of supervised vs unsupervised RT on strength outcomes
synthesised from ten studies'' could be more succinctly stated as ``A
meta-analysis of 10 studies demonstrated a moderate benefit of
supervision on strength outcomes (standardized mean effect = 0.40
{[}95\%CI: 0.06, 0.74{]}; Fisher et al., 2022).'' • Error, second
paragraph: ``Effort is conceptualized at\ldots{}'' -- ``at'' should be
``as'' • Grammar, third paragraph: ``But a caveat is that there was
limited data\ldots{}'' -- the word ``data'' is plural, so ``was'' should
be ``were.'' There are other instances of this error in the manuscript.

\hypertarget{author-response-4}{%
\subsubsection{Author Response}\label{author-response-4}}

We have however made these minor changes as noted for the existing
introduction.

\hypertarget{reviewer-comment-5}{%
\subsubsection{Reviewer Comment}\label{reviewer-comment-5}}

As with the Introduction, the Methods are thoroughly described but could
be better organized to highlight critical features. In particular, the
operational definitions of effort in this study were TUL, RPE, and RPD.
These operational definitions should be more explicitly defined and
justified as measures of effort. Currently, their description is an
afterthought buried in a paragraph: ``The TUL performed, and ratings of
perceptions of effort (RPE) and discomfort (RPD) were recorded and
compared between both conditions.'' Furthermore, whether session RPE or
RPE post each exercise is not made clear until the end of the first
paragraph in the Protocols section (``In addition, they recorded their
RPE and RPD in that order immediately upon completing the exercise using
previously validated scales for differentiating these perceptions.'')
Such critical details about the measures should be mentioned upfront in
their own paragraph.

\hypertarget{author-response-5}{%
\subsubsection{Author Response}\label{author-response-5}}

As above with the introduction revisions, we have opted not to make
further structural changes as we believe that the required information
is contained in the methods section. As noted above though we have more
clearly noted that, given the conceptualisation of effort, the exercise
performance as TUL is indicative as an operationalisation of effort in
the present study.

\hypertarget{reviewer-comment-6}{%
\subsubsection{Reviewer Comment}\label{reviewer-comment-6}}

I question the following sentence: ``This study was not pre-registered
and as described below the sample size was justified based on logistical
concerns and the analysis is considered exploratory.'' To me, it seems a
bit disingenuous to state the analysis is exploratory when there are
meta-analyses of supervision effects on RT outcomes and a clear testable
hypothesis that the study addresses using a very detailed Bayesian
analysis. Why was the experimental intervention not pre-registered? Was
this a deliberate choice? Could the study be retrospectively registered?
(I have seen others do so.)

\hypertarget{author-response-6}{%
\subsubsection{Author Response}\label{author-response-6}}

We have tried to be as transparent as possible with this study. The
original experimental study was conducted several years previously, and
before any of the authors were fully cognisant of the value of
pre-registration. Further, we did not have a sufficiently precise
statistical hypothesis to test for which we could for example have
performed a power analysis (Frequentist or Bayesian) to help justify
sample size for the experimental hypothesis. We would prefer not to
retrospectively register given this would be merely performative at this
stage and does not serve the primary purpose of pre-registration (i.e.,
the transparent reporting of research to enable the third-party
evaluation of the severity of tests for claims made based on hypothesis
tests). In retrospect we could have looked to convert the standardised
effect size metrics to the relevant scales for the parameter estimates
presented contrasting between conditions, and then conduct either an
analysis akin to examining the overlap with the region of practical
equivalence (ROPE), or calculated Bayes Factors for this. But we feel
this would be disingenuous to add (and also take more time than we have
currently) as we never planned to do this a priori. As such, having the
full dataset open and being transparent about the nature of the study is
appropriate in our opinion (notably, these responses will also be
publicly available so people can see why we made this choice).

\hypertarget{reviewer-comment-7}{%
\subsubsection{Reviewer Comment}\label{reviewer-comment-7}}

Experimental approach section: This sentence is run-on: ``This consists
of a single set of the resistance machine exercises prescribed on their
current training programme card using a load that should permit them to
reach momentary failure within a time-under-load (TUL) of 90-120 seconds
(though an upper limit of 180 seconds TUL is enforced to avoid machines
being occupied for too long on the clinic floor preventing other members
from using them) using a \textasciitilde{} 12 seconds repetition
duration (i.e., \textasciitilde4:4 seconds concentric:eccentric actions
with a 2 isometric second hold whilst still under load with tension, not
``locked-out'', on the involved musculature at the end of each
concentric and eccentric muscle action).''

\hypertarget{author-response-7}{%
\subsubsection{Author Response}\label{author-response-7}}

We have split this sentence now.

\emph{``This consists of a single set of the resistance machine
exercises prescribed on their current training programme card. A load is
used that should permit them to reach momentary failure within a
time-under-load (TUL) of 90-120 seconds (though an upper limit of 180
seconds TUL is enforced to avoid machines being occupied for too long on
the clinic floor preventing other members from using them) using a
\textasciitilde{} 12 seconds repetition duration (i.e.,
\textasciitilde4:4 seconds concentric:eccentric actions with a 2
isometric second hold whilst still under load with tension,
not''locked-out'', on the involved musculature at the end of each
concentric and eccentric muscle action).''}

\hypertarget{reviewer-comment-8}{%
\subsubsection{Reviewer Comment}\label{reviewer-comment-8}}

Previous observational sample subsection: The following sentence is run
on, contains several awkward passages, and contains punctuation errors:
``For TUL we limited data to Core and Assisted members training sessions
that either were not led by an exercise scientist or were respectively,
took the first training session after at least 6 months of previous
training at Kieser Australia had been completed by each member, randomly
sampled 1000 Core and 1000 Assisted members and then filtered to the
resistance machines used in by members in the experimental study so that
we had a selection of members across varied clinic locations and
completing sessions with a selection of resistance machine exercises and
had TUL data for each exercise.'' Suggest rewriting as a numbered list
of inclusion criteria for the TUL data.

\hypertarget{author-response-8}{%
\subsubsection{Author Response}\label{author-response-8}}

We have rewritten this as:

\emph{``For TUL we limited data to Core and Assisted members training
sessions that either were not, or were, led by an exercise scientist
respectively{[}\^{}3{]}. We took the first training session after at
least 6 months of previous training at Kieser Australia had been
completed by each member. Then we randomly sampled 1000 Core and 1000
Assisted members and filtered to the resistance machines used by members
in the experimental study.''}

\hypertarget{reviewer-comment-9}{%
\subsubsection{Reviewer Comment}\label{reviewer-comment-9}}

Protocols section:\\
• The first paragraph is overlong -- a new paragraph could begin at the
``Participants were instructed to ensure\ldots{}''\\
• Typo: ``These TUL recordings where then input\ldots{}'' -- ``where''
should be ``were''

\hypertarget{author-response-9}{%
\subsubsection{Author Response}\label{author-response-9}}

We have made both of these changes.

\hypertarget{reviewer-comment-10}{%
\subsubsection{Reviewer Comment}\label{reviewer-comment-10}}

Statistical Analysis:\\
• ``All posterior estimates and their precision,\ldots{}'' Precision
should be pluralized as ``precisions'' • Paragraph starting with ``For
both the previous observational sample\ldots{}'' The authors define
``session type'' (core or assisted) but then refer to it as
``condition'' in the model description. Suggest helping the reader by
using consistent terminology.

\hypertarget{author-response-10}{%
\subsubsection{Author Response}\label{author-response-10}}

We have pluralised ``precisions'' and also added \emph{``\ldots session
type (i.e., condition)\ldots{}''} to clarify the terminology.

\hypertarget{reviewer-comment-11}{%
\subsubsection{Reviewer Comment}\label{reviewer-comment-11}}

Figure 2 title: ``Prior Sample Data Distributions'' should be changed to
``Previous observational sample histograms''

\hypertarget{author-response-11}{%
\subsubsection{Author Response}\label{author-response-11}}

We have changed the title.

\hypertarget{reviewer-comment-12}{%
\subsubsection{Reviewer Comment}\label{reviewer-comment-12}}

An issue I sensed with the first version of the paper but failed to
articulate in my original comments was the lack of validation of the
modeling approach. The study features many assumptions and modeling
decisions, and the reader is left to wonder about the extent to which
the conclusions are robust to them. A comment on the sensitivity of the
conclusions to the modeling decisions would be appropriate in the
Discussion.

\hypertarget{author-response-12}{%
\subsubsection{Author Response}\label{author-response-12}}

Whilst we agree to some extent that all elements of model selection and
the assumptions underlying them should be well justified and explained
the standards are often double in this regard - researchers in our field
using more traditional t-test or ANOVA type analyses (and which probably
don't even understand the assumptions underlying them) don't get asked
to justify these. We've explained quite clearly, and also provided the
lay summary for our modelling choices. Adding a comment to the
discussion about the possible sensitivity of our conclusions to our
choice of models is in one sense tautological as our results, as are any
using a single modelling approach, conditional on that model (the
exception being in the case of Bayesian Model Averaging) but also that
making that point leads to the question why not take ``multiverse'' type
approach and try out lots of different models to see what happens. In
principle this would be great, but as we've noted we don't have time to
continue working on this in that fashion. If readers are curious then
the data is openly available and they are welcome to examine whether a
different, perhaps more justifiable, modelling choice would lead to
substantively different estimates.

\hypertarget{reviewer-comment-13}{%
\subsubsection{Reviewer Comment}\label{reviewer-comment-13}}

In addition, diagnostic analyses were performed and included as a
supplementary file, which I commend, but they were not mentioned in the
Results or Discussion. A quick sentence or two summarizing the
diagnostics would help assure the reader about the model validities.

\hypertarget{author-response-13}{%
\subsubsection{Author Response}\label{author-response-13}}

We have added to the analysis section where we link to the diagnostic
plots:

\emph{``Trace plots were produced along with \(\hat{R}\) values to
examine whether chains had converged (all were \(\hat{R}<1.05\)
indicating convergence)\ldots{}''}

\hypertarget{reviewer-comment-14}{%
\subsubsection{Reviewer Comment}\label{reviewer-comment-14}}

As I alluded to above, the study purpose could be more explicitly
articulated in the first paragraph of the Discussion. I suggest
modifying the first sentence as follows: ``Our study is a first to offer
insights into the role of supervision on exerciser effort during RT in
an ecologically valid real-world setting. These insights were derived
from a unique and strong study design involving observational and
experimental components.''

I suggest modifying the second sentence as follows to focus on the
outcomes measures as operational definitions of effort: ``\ldots we
assessed effort as the TUL, RPE, and RPD.''

\hypertarget{author-response-14}{%
\subsubsection{Author Response}\label{author-response-14}}

We have changed these sentence to the ones suggested.

\hypertarget{reviewer-comment-15}{%
\subsubsection{Reviewer Comment}\label{reviewer-comment-15}}

Some paragraphs are overlong. The limitations paragraph, for example,
takes up a whole page. It could be broken up into single paragraphs for
each of the \textasciitilde4 limitations discussed by the authors.

\hypertarget{author-response-15}{%
\subsubsection{Author Response}\label{author-response-15}}

We have broken up the limitations paragraph.

\hypertarget{reviewer-2---anonymous}{%
\subsection{Reviewer 2 - Anonymous}\label{reviewer-2---anonymous}}

\hypertarget{reviewer-comment-16}{%
\subsubsection{Reviewer Comment}\label{reviewer-comment-16}}

I encourage the authors to revise the conclusion so that it: 1.
Maintains the statement that under real-world settings there was little
difference in exercise performance. 2. Clearly states that the observed
benefit in the experimental setup was based on an acute effect of a
single session. 3. Avoids speculation and focuses on direct
interpretations of the data.

\hypertarget{author-response-16}{%
\subsubsection{Author Response}\label{author-response-16}}

We have made changes to the conclusions to try to more accurately
reflect the findings as suggested (and also minor change to the
abstract). The conclusion now reads:

\emph{``In real world settings there was little difference in exercise
performance suggesting similar degrees of effort during both supervised
and unsupervised conditions. Yet, in our experimental study there was a
clear benefit to performance when under supervision suggesting trainees
performed with a higher degree of effort. There was a small difference
in RPE reported with and without supervision under both real world and
experimental conditions suggesting that under supervision trainees train
with greater proximity to failure, which was also supported by greater
RPD under supervision; though in real world conditions the RPE reported
suggested that effort may be sub-optimal irrespective of condition.
These results in general support previous work highlighting the
importance of supervision during RT and that under experimental
conditions trainees likely train with greater effort when supervised.''}

\hypertarget{refs}{}
\begin{CSLReferences}{1}{0}
\leavevmode\vadjust pre{\hypertarget{ref-steeleWhatPerceptionEffort2020}{}}%
Steele, James. 2020. {``What Is (Perception of) Effort? {Objective} and
Subjective Effort During Attempted Task Performance.''} PsyArXiv.
\url{https://doi.org/10.31234/osf.io/kbyhm}.

\end{CSLReferences}



\end{document}
